\documentclass[12pt]{article}
\usepackage[UTF8]{ctex}
\usepackage{amsmath,amssymb,amsthm}
\usepackage{graphicx}
\usepackage{subfigure}
\usepackage{booktabs}
\usepackage{multirow}
\usepackage{multicol}
\usepackage{geometry}
\usepackage{listings}
\usepackage{xcolor}
\usepackage{hyperref}

\geometry{a4paper,left=2.5cm,right=2.5cm,top=2.5cm,bottom=2.5cm}

\lstset{
    language=Python,
    basicstyle=\ttfamily\small,
    keywordstyle=\color{blue},
    commentstyle=\color{gray},
    stringstyle=\color{red},
    numbers=left,
    numberstyle=\tiny\color{gray},
    stepnumber=1,
    numbersep=5pt,
    backgroundcolor=\color{white},
    showspaces=false,
    showstringspaces=false,
    showtabs=false,
    frame=single,
    tabsize=4,
    captionpos=b,
    breaklines=true,
    breakatwhitespace=false,
    escapeinside={\%*}{*)}
}

	itle{实验四:朴素贝叶斯分类器在葡萄酒质量分类中的应用}
\author{姓名:路畅通\quad 学号:2311668\quad 专业:计算机科学与技术}
\date{实验日期:\\2025年11月16日}

\begin{document}

\maketitle

\section{实验目的}
\begin{enumerate}
    \item 掌握朴素贝叶斯分类器的基本原理和数学推导。
    \item 学习使用分层采样方法划分数据集并保持类别比例稳定。
    \item 实现高斯朴素贝叶斯分类器并完成性能评估与对比。
    \item 深入理解分类器评估指标:混淆矩阵、精确率、召回率、F1分数。
    \item 掌握ROC曲线和AUC值的计算与分析方法,并探索概率校准。
    \item 探索特征工程与模型比较对朴素贝叶斯性能的影响。
\end{enumerate}

\section{实验环境}
\begin{itemize}
    \item Python 3.13(venv虚拟环境)。
    \item 主要依赖:numpy 2.1、pandas 2.2、matplotlib 3.9、seaborn 0.13、scikit-learn 1.5。
    \item 开发工具:VS Code + PowerShell 5.1 终端。
\end{itemize}

\section{数据集介绍}
\subsection{红葡萄酒数据集特征}
数据集包含1599个红葡萄酒样本,11个物理化学特征:固定酸度、挥发性酸度、柠檬酸、残糖、氯化物、游离二氧化硫、总二氧化硫、密度、pH值、硫酸盐、酒精含量。目标变量为质量评分(3-8分)。

\subsection{数据分布统计}
\begin{table}[htbp]
\centering
\caption{葡萄酒质量三分类分布(低/中/高)}
\begin{tabular}{lcccc}
\toprule
质量类别 & 原始数量 & 训练集数量 & 测试集数量 & 训练占比(\%) \\
\midrule
低质量(0) & 63 & 44 & 19 & 69.84 \\
中等质量(1) & 1319 & 923 & 396 & 69.98 \\
高质量(2) & 217 & 152 & 65 & 70.05 \\
\bottomrule
\end{tabular}
\end{table}

类别映射:质量分数3-4为低,5-6为中,7-8为高。分层抽样保持了各类别约70\%/30\%的比例。

\section{实验内容与步骤}

\subsection{实验基本要求:数据集划分与朴素贝叶斯分类器实现}

\subsubsection{实验内容}
\begin{enumerate}
    \item \textbf{数据预处理与特征工程}:检查缺失值、标准化11个特征、重新编码多分类标签。
    \item \textbf{分层采样划分}:按70\%/30\%划分训练与测试集,并保持类别比例。
    \item \textbf{朴素贝叶斯实现}:手写高斯朴素贝叶斯(先验与条件概率使用对数形式)并与scikit-learn版本对比。
    \item \textbf{基础评估}:记录两种实现的准确率,输出混淆矩阵图。
\end{enumerate}

\subsubsection{关键代码实现}
\begin{lstlisting}[caption=分层采样与朴素贝叶斯分类器实现,label=lst:basic]
def map_quality(q: int) -> int:
    if q <= 4:
        return 0
    elif q <= 6:
        return 1
    return 2

df = pd.read_csv(DATA_PATH)
y = df["quality"].apply(map_quality).values
X = df.drop(columns=["quality"]).values

X_train, X_test, y_train, y_test = train_test_split(
    X, y, test_size=0.3, stratify=y, random_state=42
)

class ManualGaussianNB:
    def fit(self, X, y):
        self.classes_, counts = np.unique(y, return_counts=True)
        self.theta_ = np.vstack([X[y == c].mean(axis=0) for c in self.classes_])
        self.var_ = np.vstack([X[y == c].var(axis=0) + 1e-9 for c in self.classes_])
        self.log_prior_ = np.log((counts + 1) / (counts.sum() + len(self.classes_)))
        return self

    def predict(self, X):
        log_prob = []
        for mean, var, log_prior in zip(self.theta_, self.var_, self.log_prior_):
            lp = -0.5 * np.log(2 * np.pi * var) - (X - mean) ** 2 / (2 * var)
            log_prob.append(lp.sum(axis=1) + log_prior)
        return self.classes_[np.argmax(np.vstack(log_prob).T, axis=1)]

manual_nb = ManualGaussianNB().fit(X_train, y_train)
y_pred_manual = manual_nb.predict(X_test)
sk_nb = GaussianNB().fit(X_train, y_train)
y_pred_sk = sk_nb.predict(X_test)
print("手写/库实现准确率:", accuracy_score(y_test, y_pred_manual),
      accuracy_score(y_test, y_pred_sk))
\end{lstlisting}

\subsection{实验中级要求:分类器性能深入评估}

\subsubsection{实验内容}
\begin{enumerate}
    \item \textbf{多维度性能评估}:利用\texttt{classification\_report}计算每类精确率、召回率、F1以及宏/加权平均。
    \item \textbf{可视化分析}:输出热力图混淆矩阵与特征重要性条形图。
    \item \textbf{特征重要性}:通过ANOVA F检验衡量特征的判别力,辅助选择。
\end{enumerate}

\subsubsection{关键代码实现}
\begin{lstlisting}[caption=性能评估与特征分析,label=lst:intermediate]
clf = GaussianNB().fit(X_train, y_train)
y_pred = clf.predict(X_test)
report = classification_report(y_test, y_pred, digits=4)
with open("outputs/intermediate/classification_report.txt", "w", encoding="utf-8") as f:
    f.write(report)

cm = confusion_matrix(y_test, y_pred)
plt.figure(figsize=(6, 5))
sns.heatmap(cm, annot=True, fmt="d", cmap="Blues",
            xticklabels=["低","中","高"], yticklabels=["低","中","高"])
plt.title("朴素贝叶斯混淆矩阵")
plt.savefig("confusion_matrix_custom.png", dpi=150)

scores, _ = f_classif(X_train, y_train)
order = np.argsort(scores)[::-1]
plt.figure(figsize=(8, 5))
sns.barplot(x=scores[order][:10], y=np.array(feature_names)[order][:10], orient="h")
plt.xlabel("ANOVA F 值")
plt.savefig("feature_importance.png", dpi=150)
\end{lstlisting}

\subsection{实验高级要求:ROC曲线与AUC分析}

\subsubsection{实验内容}
\begin{enumerate}
    \item \textbf{多类别ROC}:One-vs-Rest策略绘制各类别曲线、宏/微平均曲线并计算AUC。
    \item \textbf{概率校准}:使用Platt Scaling(\texttt{CalibratedClassifierCV})输出校准曲线比较可靠性。
    \item \textbf{模型比较}:对比朴素贝叶斯、逻辑回归、随机森林、SVC的宏平均ROC表现。
\end{enumerate}

\subsubsection{关键代码实现}
\begin{lstlisting}[caption=ROC曲线与AUC分析,label=lst:advanced]
nb = GaussianNB().fit(X_train, y_train)
y_score = nb.predict_proba(X_test)
y_bin = label_binarize(y_test, classes=[0, 1, 2])

fpr, tpr, roc_auc = {}, {}, {}
for i in range(3):
    fpr[i], tpr[i], _ = roc_curve(y_bin[:, i], y_score[:, i])
    roc_auc[i] = auc(fpr[i], tpr[i])

calib = CalibratedClassifierCV(estimator=nb, method="sigmoid", cv=5)
calib.fit(X_train, y_train)
for cls in range(3):
    true_bin = (y_test == cls).astype(int)
    frac_pos, mean_pred = calibration_curve(true_bin, y_score[:, cls], n_bins=10)
    frac_pos_cal, mean_pred_cal = calibration_curve(true_bin,
                                                   calib.predict_proba(X_test)[:, cls],
                                                   n_bins=10)

models = {
    "朴素贝叶斯": GaussianNB(),
    "逻辑回归": LogisticRegression(max_iter=1000, random_state=42),
    "随机森林": RandomForestClassifier(n_estimators=200, random_state=42),
    "SVC": SVC(probability=True, random_state=42)
}

def compute_macro_auc(y_true_bin, proba):
    fpr_list, tpr_list = [], []
    for i in range(y_true_bin.shape[1]):
        fpr_i, tpr_i, _ = roc_curve(y_true_bin[:, i], proba[:, i])
        fpr_list.append(fpr_i)
        tpr_list.append(tpr_i)
    all_fpr = np.unique(np.concatenate(fpr_list))
    mean_tpr = np.zeros_like(all_fpr)
    for fpr_i, tpr_i in zip(fpr_list, tpr_list):
        mean_tpr += np.interp(all_fpr, fpr_i, tpr_i)
    mean_tpr /= y_true_bin.shape[1]
    return auc(all_fpr, mean_tpr)

for name, model in models.items():
    model.fit(X_train, y_train)
    proba = model.predict_proba(X_test)
    macro_auc = compute_macro_auc(y_bin, proba)
    print(name, macro_auc)
\end{lstlisting}

\section{实验结果与分析}

\subsection{数据预处理结果}
分层抽样保持了原始比例(见表\ref{tab:split})。训练集中低/中/高样本分别为44/923/152,对应测试集19/396/65。由于特征无缺失值,因此直接进行标准化处理。

\begin{table}[htbp]
\centering
\caption{分层采样后数据集分布}
\label{tab:split}
\begin{tabular}{lcccc}
\toprule
质量类别 & 原始数量 & 训练集数量 & 测试集数量 & 训练占比(\%) \\
\midrule
低质量(0) & 63 & 44 & 19 & 69.84 \\
中等质量(1) & 1319 & 923 & 396 & 69.98 \\
高质量(2) & 217 & 152 & 65 & 70.05 \\
总计 & 1599 & 1119 & 480 & 70.00 \\
\bottomrule
\end{tabular}
\end{table}

\subsection{分类器性能评估}
手写与scikit-learn版高斯朴素贝叶斯在测试集上均取得0.7792的准确率。详细指标见表\ref{tab:metrics},对应混淆矩阵图如图\ref{fig:confusion_matrix}。

\begin{table}[htbp]
\centering
\caption{朴素贝叶斯分类器详细性能指标}
\label{tab:metrics}
\begin{tabular}{lccccc}
\toprule
类别 & 精确率 & 召回率 & F1分数 & 支持度 & AUC值 \\
\midrule
低质量(0) & 0.2500 & 0.2632 & 0.2564 & 19 & 0.7421 \\
中等质量(1) & 0.9076 & 0.8182 & 0.8606 & 396 & 0.7488 \\
高质量(2) & 0.4369 & 0.6923 & 0.5357 & 65 & 0.8383 \\
\midrule
宏平均 & 0.5315 & 0.5912 & 0.5509 & -- & 0.7788 \\
加权平均 & 0.8178 & 0.7792 & 0.7927 & -- & 0.9069(微) \\
\bottomrule
\end{tabular}
\end{table}

\begin{figure}[htbp]
\centering
\includegraphics[width=0.8\textwidth]{confusion_matrix_custom.png}
\caption{朴素贝叶斯分类器混淆矩阵。模型主要集中预测中等质量类别,低/高质量样本存在混淆。}
\label{fig:confusion_matrix}
\end{figure}

\subsection{特征重要性分析}
ANOVA F值排名显示酒精含量、挥发性酸度、硫酸盐对分类贡献最大(见图\ref{fig:feature_importance})。Top-5特征的F值得分如下:

\begin{table}[htbp]
\centering
\caption{特征重要性(F检验)前五名}
\begin{tabular}{lc}
\toprule
特征 & F值 \\
\midrule
酒精(alcohol) & 107.83 \\
挥发性酸度(volatile acidity) & 64.76 \\
柠檬酸(citric acid) & 31.63 \\
硫酸盐(sulphates) & 23.42 \\
总二氧化硫(total sulfur dioxide) & 20.41 \\
\bottomrule
\end{tabular}
\end{table}

\begin{figure}[htbp]
\centering
\includegraphics[width=0.9\textwidth]{feature_importance.png}
\caption{特征重要性可视化。酒精含量的F值远高于其他特征,表明其与质量标签最相关。}
\label{fig:feature_importance}
\end{figure}

\subsection{ROC曲线与AUC分析}
One-vs-Rest ROC 曲线如图\ref{fig:roc_curves} 所示,高质量类别(2)拥有最高AUC 0.8383,说明对高质量样本的区分度更高。宏/微平均AUC分别为0.7788和0.9070。

\begin{figure}[htbp]
\centering
\includegraphics[width=0.9\textwidth]{roc_curves_custom.png}
\caption{多类别ROC曲线。虚线为宏/微平均。}
\label{fig:roc_curves}
\end{figure}

概率校准曲线如图\ref{fig:calibration}。Platt校准后每个类别的曲线更贴近对角线,预测概率更可信。

\begin{figure}[htbp]
\centering
\includegraphics[width=0.9\textwidth]{calibration_curves.png}
\caption{概率校准曲线,比较校准前后概率的可靠性。}
\label{fig:calibration}
\end{figure}

\subsection{模型综合比较}
使用相同特征和数据划分,比较四种模型的准确率、宏平均F1、加权F1及宏平均AUC,结果见表\ref{tab:modelcompare} 与 图\ref{fig:model_comparison_roc}。

\begin{table}[htbp]
\centering
\caption{不同分类器性能比较}
\label{tab:modelcompare}
\begin{tabular}{lcccc}
\toprule
分类器 & 准确率 & 宏平均F1 & 加权平均F1 & 宏平均AUC \\
\midrule
朴素贝叶斯 & 0.7792 & 0.5509 & 0.7927 & 0.7788 \\
逻辑回归 & 0.8396 & 0.4550 & 0.8108 & 0.8233 \\
随机森林 & 0.8771 & 0.5317 & 0.8561 & 0.8868 \\
支持向量机 & 0.8417 & 0.4484 & 0.8094 & 0.8295 \\
\bottomrule
\end{tabular}
\end{table}

\begin{figure}[htbp]
\centering
\includegraphics[width=0.9\textwidth]{model_comparison_roc.png}
\caption{多模型宏平均ROC曲线比较,随机森林在各阈值下保持最高TPR。}
\label{fig:model_comparison_roc}
\end{figure}

\section{实验总结与讨论}
\subsection{总结}
\begin{itemize}
    \item 手写高斯朴素贝叶斯与\texttt{GaussianNB}在数值上完全一致,验证了对先验与条件概率实现的正确性。
    \item 数据高度不平衡(中等质量样本占82.5\%),导致模型偏向预测主类别,低/高类别召回率不足,需要进一步的采样或代价敏感策略。
    \item 特征工程显示酒精、挥发性酸度、硫酸盐对质量区分贡献最大,后续可针对这些特征进行更细致的业务分析。
    \item 随机森林在准确率与宏平均AUC上显著优于朴素贝叶斯,说明非线性模型能更好地捕捉复杂特征交互。
\end{itemize}

\subsection{心得体会}
\begin{itemize}
    \item 通过从零实现朴素贝叶斯,更加理解了对数似然、方差平滑等细节对稳定性的影响。
    \item 实验确认了中文绘图需要显式设置字体,否则matplotlib默认字体缺少CJK字形;统一在脚本中添加`SimHei`设置后问题解决。
    \item ROC与概率校准的组合分析提供了比单一准确率更全面的视角,可帮助识别低概率预测的可信度。
    \item 模型比较环节表明:在保持同一特征工程的前提下,适度引入集成模型或调参即可获得明显收益,这为后续优化提供了方向。
\end{itemize}

\end{document}
